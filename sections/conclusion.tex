\section{Discussion and Conclusion}
\label{sec:discuss} 


\textbf{Context-dependent perturbation.}
While the LM-based perturbation is more effective to generate, it has limited support for more complex tasks.
For example, \citet{gardner2020contrast} created contrast sets for multi-hop question answerin (DROP~\cite{}) by adding compositional reasoning steps \emph{related to} the corresponding paragraph.
Such context-aware perturbation is not achievable when the input to the LM model is just the question sentence.
Rather, one should consider training models that take additional contexts as part of the input.

\textbf{Grouped labeling and active learning.}
While we labeled perturbations individually, it is also possible to make it more efficient through grouped labeling. 
If we categorize generated sentences based on their perturbations types or template patterns (similar to the featurization in \S\ref{subsec:global_exp}), we can create labeling functions~\cite{ratner2017snorkel} for weak supervision based on different combinations of perturbation metadata.

%{Or, a more sophisticated way to compute P is to create snorkel labeling functions based on different combinations of perturbation metadata.}

\textbf{Counterfactual explanation.}
Arguably, perturbations can also be summarized into SHAP-like feature attributions.
However, it is unclear if the perturbations induce appropriate amount of information for feature attribution.
In a simple \exinline{this is a good movie}, changing \swap{good}{great} does not affect the prediction, yet it should not be interpreted as ``good'' being trivial.
Moreover, because our perturbations are not exhaustive, the summarized explanation may be biases by the changed-to words.
To effectively support feature attribution, future work can explore more nuanced, semantic ``expectation'' of certain changes, \ie changing words to antonyms may not affect the label, but changing it to synonyms may not.



%\textbf{Other applications.}
%Beyond our demonstrated applications, \sysname can potentially support other use cases. 
%For example, grouping generated sentences based on their perturbations types or template patterns, one can potentially 
