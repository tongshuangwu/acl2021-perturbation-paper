
\section{Ethical Consideration}
\wts{This is an extra discussion section that's recommended, and can go beyond 8 pages: https://2021.aclweb.org/ethics/Ethics-FAQ/}

Our work includes crowdlabeling counterfactuals, whose detailed procedures are in Appendix~\ref{appendix:label_instruct}.
The base tasks and datasets are standard and do not include sensitive or personal information, and therefore we do not foresee ethical issues. 
As for model training, the most expensive experiment is on finetuning GPT-2 into \sysname, which took around 8 hours (Appendix~\ref{appendix:train_data}).
To reduce the cost, we may consider optimizing the finetuning dataset to reduce repetitive senten pairs, and thereby train \sysname with fewer shots.

More critically, \sysname may be misused in the semi-automatic settings. 
As discussed in \S\ref{sec:discuss}, \sysname inherits some biases from its training datasets.
However, as it generates multiple counterfactuals, humans may assume that \sysname comprehensively cover \emph{all} the counterfactuals possible, and develop over-reliance~\cite{bansal2021does}.
The over-reliance may hinder humans' counterfactual coverage during labeling or analysis.
We plan to highlight the limitations along with the model deployment. 
In future work, we also plan to design interactive interfaces that help humans to properly collaborate with \sysname.
