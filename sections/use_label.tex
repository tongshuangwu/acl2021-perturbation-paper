\newcommand{\maug}{\texttt{m-polyjuice}\xspace}
\newcommand{\mcomp}{\texttt{m-baseline}\xspace}
\newcommand{\mcad}{\texttt{m-CAD}\xspace}

\definecolor{ccon}{HTML}{fee9d4}
\definecolor{cood}{HTML}{d8f0d3}
\definecolor{cid}{HTML}{dae8f5}


\newcommand{\TableContrastSet}{
\begin{table}
\small
\centering
\setlength{\tabcolsep}{3pt}
\begin{tabular}{@{} c c c l c @{}}
\toprule
\textbf{Task} & \textbf{Dev.} & \textbf{Orig. set} & \textbf{Contrast set} $\downarrow$ & \textbf{Consistency} $\downarrow$ \\ 
\midrule
\sst & 94.3 & 93.8 & 84.9 (-8.9) & 76.1 \\
\nli & 86.5 & 91.6 & 72.3 (-19.3) & 56.4 \\
\qqp & 91.7 & 87.5 & 75.3 (-12.2) & 61.1\\
% \sst-distilbert & 91.1 & 93.2 & 84.0 (-9.2) & 75.0 \\
% \sst-bert & 92.4 & 90.9 & 85.8 (-9.2) & 76.1 \\
% \sst-roberta & 94.3 & 93.8 & 84.9 (-8.9) & 76.1 \\

% \nli-bert & 78.6 & 83.1 & 69.7 (-13.4) & 49.5 \\
% \nli-roberta & 86.5 & 91.6 & 72.3 (-19.3) & 56.4 \\
% \qqp-bert & 90.9 & 88.1 & 75.7 (-12.4) & 60.5 \\
% \qqp-distilbert & 89.7 & 88.1 & 72.4 & 57.3\\
% \qqp-roberta & 91.7 & 87.5 & 75.3 & 61.1\\
% the baseline
% \nli & 86.5 & 80.6 & 78.6 (-19.3) & 30.4 \\
% using a imdb model
% imdb_contrast_test 96.7 / 89.1 / 86.1
% imdb_iclr_test 96.7 / 91.0 / 87.9
% using a sst-2 model
% imdb_iclr_test 91.3 / 89.5 / 81.4
% imdb_iclr_test 89.5 / 87.3 / 77.3
\bottomrule
\end{tabular}
\vspace{-5pt}
\caption{\sysname $\xp$ as contrasts sets, with model accuracy on the development set, the original set of $x$, the contrast sets, and consistency (cases where the model predicts both $x$ and $\xp$ correctly).
The numbers drop for a similar amount on \sysname contrast sets compared to expert-created sets~\cite{gardner2020contrast}, on which the accuracy of all classification models decreases by $9.8$ on average, with a consistency of ${\approx}64.1$.
This indicates \sysname can also reveal model insufficiencies.
%\ie with similar drop in performance (on average $-9.8$ for all classification tasks in \cite{gardner2020contrast}) and consistency (${\approx}64.1$ in manual contrast set).
%The model performance here drops for a similar degree compared to the manually created contrast sets~\cite{gardner2020contrast} ($-9.78$ for all )
}
%\vspace{-5pt}
\label{table:contrast_set_result}
\end{table}
}
%\end{comment}

\newcommand{\TableAugSST}{
\begin{table*}
\small
\centering
\setlength{\tabcolsep}{5pt}
\begin{tabular}{@{}lllllllll@{}}
\toprule
 Model & 
 \cellcolor{cid}SST-2 & 
 \cellcolor{cood}Senti140 & 
 \cellcolor{cood}SemEval & 
 \cellcolor{cood}Amzbook & 
 \cellcolor{cood}Yelp & 
 \cellcolor{cood}IMDB & 
 \cellcolor{ccon}IMDB-Cont. & 
 \cellcolor{ccon}{IMDB-CAD} \\
\midrule
\mcomp & $92.9\pm 0.2$ & $88.9\pm 0.3$ & $84.8\pm 0.5$ & $85.1\pm 0.4$ & $90.0\pm 0.3$ & $90.8\pm 0.5$ & $92.2\pm 0.6$ & $86.5\pm 0.2$ \\
\maug	 & $92.7\pm 0.2$ & $\mathbf{90.7\pm 0.4}$ & $\mathbf{86.4\pm 0.1}$ & $85.6\pm 0.8$ & $90.1\pm 0.0$ & $90.6\pm 0.3$ & $\mathbf{94.0\pm 0.3}$ & $\mathbf{89.7\pm 0.5}$ \\
 % imdb_contrast_test: 91.1 (9.4) / 92.8 (0.4)
 % imdb_contrast_test: 87.4 (0.0) / 89.6 (0.5)
 % imdb_iclr_test 93.0 (0.3) / 93.9 (0.4)
 % imdb_iclr_dev 92.0 (0.2) / 92.7 (0.2)
\bottomrule
\end{tabular}
\vspace{-5pt}
\caption{\sst model performance, with $n{=}4,000$ and $m{=}2,000$.
\textbf{Bolded} cells highlight significant improvements.
\maug maintains the \colbox{cid}{in-domain} and \colbox{cood}{out-of-domain} accuracies on reviews (SST-2, Amzbook, Yelp, IMDb Movie Review~\cite{ni2019justifying, asghar2016yelp, maas2011learning}), but improves on Twitter data (Senti140 and SemEval 2017~\cite{go2009twitter, rosenthal2017semeval}), likely because their distributions are less similar to SST-2 than the reviews.
The model also improves on the \colbox{ccon}{contrast sets}~\cite{gardner2020contrast, kaushik2019learning}).
% we need to say it's cheaper
%on \colbox{cid}{in domain}, \colbox{cood}{out of domain}, and \colbox{ccon}{contrast sets}. \maug performs better than \mcomp on twitter datasets (Senti140~\cite{go2009twitter}, SemEval 2017~\cite{rosenthal2017semeval}) and contrast sets IMDb-Contrast~\cite{gardner2020contrast} and IMDb-CAD~\cite{kaushik2019learning}, while maintaining the ones on reviews (SST-2, Amzbook~\cite{ni2019justifying}, Yelp~\cite{asghar2016yelp}, IMDb Movie Review~\cite{maas2011learning}).
}
%\vspace{-5pt}
\label{table:aug_sst}
\end{table*}}

%%%%%%%%%%%%%%%%%%%%%%%%%%%%%%%%%%%%%%%%%%%%%%%
\newcommand{\TableAugNLI}{
\begin{table*}
\small
\centering
\setlength{\tabcolsep}{5.7pt}
\begin{tabular}{@{}lllllllll@{}}
\toprule
 Model & 
 \cellcolor{cid}SNLI & 
 \cellcolor{cood}MNLI-m & 
 \cellcolor{cood}MNLI-mm & 
 \cellcolor{ccon}SNLI-CAD & 
 \cellcolor{ccon}break & 
 \cellcolor{ccon}DNC & 
 \cellcolor{ccon}stress & 
 \cellcolor{ccon}diagnostic \\
\midrule
\mcomp 	& $85.7\pm 0.4$& $86.1\pm 0.2$& $\mathbf{86.6\pm 0.2}$& $72.8\pm 0.3$& $86.4\pm 1.5$& $54.5\pm 0.6$& $65.1\pm 0.6$& $56.0\pm 0.8$\\
 
\mcad & $85.8\pm 0.6$ & $\mathbf{86.6\pm 0.1}$ & $85.6\pm 0.3$ & $\mathbf{73.8\pm 0.2}$ & $89.4\pm 2.9$ & $55.8\pm 0.9$ & $65.5\pm 0.5$ & $56.4\pm 0.4$ \\
 
\maug	& $85.3\pm 0.3$& $86.0\pm 0.1$& $\mathbf{86.4\pm 0.0}$& $\mathbf{73.6\pm 0.2}$& $\mathbf{89.1\pm 1.2}$& $\mathbf{57.7\pm 0.3}$& $65.1\pm 0.2$& $\mathbf{57.5\pm 0.5}$\\
%  \texttt{m-polyjuice-rand} & $85.7\pm 0.4$ & $86.1\pm 0.1$ & $86.2\pm 0.1$ & $73.4\pm 0.5$ & $87.2\pm 0.6$ & $54.7\pm 0.3$ & $64.6\pm 0.6$ & $56.9\pm 0.8$ \\

 %10000 & 1574 & comp & $85.3\pm 0.5$& $85.2\pm 0.2$& $85.4\pm 0.3$& $72.4\pm 0.1$& $86.1\pm 1.8$& $54.2\pm 1.8$& $64.0\pm 0.4$& $56.0\pm 0.3$\\
 %10000 & 1574 & aug\_gpt & $85.3\pm 0.3$& $85.0\pm 0.2$& $85.1\pm 0.1$& $73.4\pm 0.5$& $90.5\pm 1.1$& $56.5\pm 1.2$& $64.6\pm 0.5$& $57.0\pm 0.4$\\
\bottomrule
\end{tabular}
\vspace{-5pt}
\caption{\nli models, with $n{=}20,000$ and $m{=}1,574$. 
\maug improves on \colbox{ccon}{contrast} and \colbox{ccon}{challenge sets}~\cite{kim2019probing, naik2018stress, glockner-etal-2018-breaking, wang2018glue}; it gets compatible (or better) gains than \mcad (trained using manual counterfactuals), but with less implementation and annotation effort.
}
%\vspace{-10pt}
\label{table:aug_nli}
\end{table*}
}

%%%%%%%%%%%%%%%%%%%%%%%%%%%%%%%%%%%%%%%%%%%%%%%
\newcommand{\TableAugQQP}{
\begin{table}
\small
\centering
%\setlength{\tabcolsep}{5.7pt}
\begin{tabular}{lll}
\toprule
 Model & \cellcolor{cid}QQP & \cellcolor{ccon}PAWS-QQP \\
\midrule
\mcomp 	& $84.5 \pm 0.6$ & $37.0\pm 0.5$\\
 %20,000 & 1,574 & \texttt{aug-r} & $85.7\pm 0.4$ & $86.1\pm 0.1$ & $86.2\pm 0.1$ & $73.4\pm 0.5$ & $87.2\pm 0.6$ & $54.7\pm 0.3$ & $64.6\pm 0.6$ & $56.9\pm 0.8$ \\
\maug	& $84.7 \pm 1.0$ & $\mathbf{38.7\pm 0.4}$\\
 
 %10000 & 1574 & comp & $85.3\pm 0.5$& $85.2\pm 0.2$& $85.4\pm 0.3$& $72.4\pm 0.1$& $86.1\pm 1.8$& $54.2\pm 1.8$& $64.0\pm 0.4$& $56.0\pm 0.3$\\
 %10000 & 1574 & aug\_gpt & $85.3\pm 0.3$& $85.0\pm 0.2$& $85.1\pm 0.1$& $73.4\pm 0.5$& $90.5\pm 1.1$& $56.5\pm 1.2$& $64.6\pm 0.5$& $57.0\pm 0.4$\\
\bottomrule
\end{tabular}
%\vspace{-5pt}
\caption{\sysname in \qqp improves on \colbox{ccon}{PAWS}~\cite{zhang2019paws}, with $n{=}20,000$ and $m{=}1,911$.}
\vspace{-5pt}
\label{table:aug_qqp}
\end{table}
}

%%%%%%%%%%%%%%%%%%%%%%%%%%%%%%%%%%%%%%%%%%%%%%%
\newcommand{\TableAugQQPChecklist}{
\begin{table}
\small
\centering
\setlength{\tabcolsep}{4pt}
\begin{tabular}{@{}p{0.4\textwidth} r@{}}
\toprule
TESTNAME & $\Delta$ fail\% \\
\midrule
 Order does not matter for symmetric relations & -18.4\% \\
 Order does not matter for comparison & -26.5\% \\
 Order does matter for asymmetric relations & -14.5\% \\
%\midrule
% Is it \{ok, bad,..\} to \{smoke, do,..\} \{\emph{before $\not\eq$ after}\} & -52.5\% \\
% %What was person's life \{\emph{before $\not\eq$ after}\} becoming X & -46.6\% \\
% Do you have to X your dog \{\emph{before $\not\eq$ after}\} Y it & -35.4\% \\
%\midrule
% Is person X $\not\eq$ Is person becoming X & -8.5\% \\
% Is person X $\not\eq$ Did person use to be X & -5.4\% \\
\midrule
 How can I become \{\emph{more X $\not\eq$ less X}\} & -30.7\% \\
 How can I become \{\emph{more X $=$ less antonym(X)}\} & 28.0\% \\
 How can I become \{\emph{X $\not\eq$ not X}\} & -10.4\% \\
 How can I become \{\emph{X $\not\eq$ not antonym(X)}\} & -5.5\% \\
 %\midrule
 %traditional SRL: wrong active / passive swap & 2.2\% \\
 %traditional SRL: active / passive swap with people & -6.4\% \\
 %traditional SRL: active / passive swap & -15.2\% \\
%\midrule
 %Change first and last name in one of the questions & -11.5\% \\
 %(q, paraphrase(q)) & -5.3\% \\
\bottomrule
\end{tabular}
\vspace{-5pt}
\caption{
Sample \qqp CheckList tests, with $\Delta$fail\% denoting the failure rate change from \mcomp to \maug. 
%However, the model gets significantly better on \texttt{more X $\not\eq$ less X} by sacrificing \texttt{more X $=$ less antonym(X)}.
With $n=20,000$ and $m=1,911$, \maug failed less on 11 tests (out of the 27 where \mcomp failed), while only becoming worse on 2.
%Meanwhile, the models have similar accuracies on the test set ($84.5 \pm 0.6$ for \maug, and $84.7 \pm 1.0$ for \mcomp).
%Some sample tests are in Table~\ref{table:aug_qqp}.
The model improves consistently on most related cases, but possibly overfits on \texttt{more/less}. 
}
\label{table:aug_qqp}
\vspace{-10pt}
\end{table}}



%%%%%%%%%%%%%%%%%%%%%%%%%%%%%%%%%%%%%%%%%%%%%%%
\section{Counterfactual Evaluation \& Training}
\label{sec:app_label}

We ask crowdworkers to label \sysname-generated counterfactuals for \sst, \nli, and \qqp, for the purposes of evaluation and training.\footnote{We collect \emph{asymmetric counterfactuals}~\cite{garg2019counterfactual} by sampling more \emph{Duplicate} and \emph{Entailment} examples in \qqp and \nli to perturb, due to the difficulty of flipping other labels.} 
In each labeling round, the worker is presented with an original $x$ and its label, and asked to annotate the groundtruth for three $\xp$, rejecting non-fluent ones (details and interface in Appendix~\ref{appendix:label_instruct}).

We use a simple heuristic to select which counterfactuals are presented for labeling, aimed at increasing diversity. 
Having each $\xp$ be represented by its token changes, control code, and dependency tree structure, we greedily select the ones that are least similar to those already selected for labeling. 
This avoids redundancy in the labeling set, \eg common perturbation patterns such as \swap{black}{white}.
%, or counterfactuals that repeatedly change one token, \eg \remove{black} to \add{blue}, \add{red}, etc.
% being applied to many original $x$.
% Manually created counterfactuals are useful for evaluating and improving models~\cite{gardner2020contrast,kaushik2019learning}, but are expensive.
% To verify that \sysname helps achieve the same purposes, we task crowdworkers to \emph{only label} automatically generated counterfactuals for \sst, \nli, and \qqp (the same as in \S\ref{subsec:intrinsic}).
% For each labeling round, based on the original $x$ and its groundtruth\footnote{For \qqp and \nli, we only perturbed \emph{duplicate} and \emph{entailment} examples, as others are significantly harder to flip (called \emph{asymmetric counterfactuals}~\cite{garg2019counterfactual}).}, the annotator labels three $\xp$ by (1) the class label and (2) fluency.
% We remove noisy workers as well as noisy labels (details in Appendix~\ref{appendix:label_instruct}), and use the collected data for the evaluation (\S\ref{subsec:contrast_set}) and training experiments (\S\ref{subsec:augmentation}).

% \subsection{Selecting diverse set to label}
% \label{subsec:gen_counterfactual_for_labeling}

%Not all counterfactuals have the same labeling utility, and we use the following two strategies to better compensate for the training space.

% \paragraph{Selecting diverse counterfactuals.}

%Starting with a set of $\xset$, we create a set of counterfactuals $\mathbf{L}$ for the crowdworker to label.
%For each $x_i \in \xset$, we generate a large set of $\hat{\xset}_i$, and sample three of $\xp \in \hat{\xset}_i$ to $\mathbf{L}$, such that crowdworkers label the same numbers of variations for each $x$.
% To improve coverage around decision boundaries, we further enhance the \emph{diversity} of the selected subset through submodular optimization.
% The selection uses the relationship $\relation{\xp}$, which involves: \{\ $x$, tokens \remove{removed} or \add{added}, the corresponding \tagstr and dependency structure \ \}
% We define a distance function $D_L(\relation{\xp_i}, \relation{\xp_j})$, which is a weighted combination of the distances between each individual component in $\relation{\xp}$.
% We then greedily select $\xp$ whose $\relation{\xp}$ is the least similar to those already in the counterfactual set to label.
% For example, if \exinline{a \swap{man}{woman} walks} is already in the set, we would penalize \exinline{the \swap{man}{woman} is dancing} (same changed text) or \exinline{he's with the \swap{man}{girl}} (\ctrltag{lexical} change), but prioritize \exinline{\add{two} people talk.}



%The similarity computation is in Appendix~\ref{appendix:perturb_similarity}. 
%\wts{Need to write this part.} 
%Formally, the distance between two counterfactuals is ($a_1$ is an abbreviation for $a(\xp_1, x)$):
%$$d(\xp_1, \xp_2) = \alpha\cdot\mathbb{1}(s_1 = s_2) + \beta\cdot\mathbb{1}(r_1 = r_2) + \gamma\cdot\mathbb{1}(a_1=a_2)$$
%With $\gamma = \beta > \alpha$ (empirically $2/5$, $2/5$, $1/5$).



\subsection{Evaluation with Contrast Sets}
\label{subsec:contrast_set}

We verify whether \sysname counterfactuals can be used to create \emph{contrast sets}~\cite{gardner2020contrast}, \ie evaluation sets where each instance has a nearby counterfactual with a \emph{different} groundtruth, to better evaluate model decision boundaries.
We construct these sets by simply filtering out counterfactuals that are labeled the same as their original instances (40\%--63\% depending on the task).

For each task, we test multiple classifers open-sourced by Huggingface~\cite{Wolf2019HuggingFacesTS}, and report the best performing model for each\footnote{\UrlFont{huggingface.co/\{roberta-large-mnli, textattack/roberta-base-SST-2, ji-xin/roberta\_base-QQP-two\_stage\}}} in Table~\ref{table:contrast_set_result} (results for other models are analogous). \sysname contrast sets display performance gaps consistent with those of \citet{gardner2020contrast}, where the sets are constructed manually by NLP researchers.

\TableContrastSet

% Do \sysname counterfactuals serve as a useful test set for uncovering model errors?
% \citet{gardner2020contrast} created \emph{contrast sets}, \ie minimally edited instances for inspecting model deficiencies, which we try to replicate.
% Following the definition, we filter the counterfactuals to only keep those whose groundtruth label is different from $x$'s, resulting in contrast sets with sizes 100--300.
%: 106 $\xp$ on 88 $x$ for \sst, 276 $\xp$ on 202 $x$ for \nli, 243 $\xp$ on 185 $x$ for \qqp.
%We changed the \nli labels for 60\% of the time, whereas \sst was harder (only flipping 36.9\%).

%\paragraph{Models \& results.}
% We test finetuned BERT~\cite{devlin-etal-2019-bert}, RoBERTa~\cite{liu2019roberta}, and DistilBERT~\cite{Sanh2019DistilBERTAD} models opensourced on Huggingface~\cite{Wolf2019HuggingFacesTS}, and report the best performing models on the validation set, which happen to be the RoBERTa model across tasks.\footnote{
% \UrlFont{huggingface.co/\{roberta-large-mnli, textattack/roberta-base-SST-2, ji-xin/roberta\_base-QQP-two\_stage\}}}
%We report model accuracies on the full validation sets, the original examples for collecting counterfactuals, and the contrast sets.
%We also report consistency, \ie cases where the model predicts both the originals and the counterfactuals correctly.
% Table~\ref{table:contrast_set_result} shows that all state-of-the-art models perform significantly worse on our contrast sets, and the performance decreases for a similar amount as in~\cite{gardner2020contrast}.
% Results on other unreported models are consistent.
% In other words, \sysname counterfactuals can reveal models' limited capabilities as the original contrast sets.


%%%%%%%%%%%%%%%%%%%%%
\TableAugSST
\TableAugNLI
\TableAugQQP
%%%%%%%%%%%%%%%%%%%%%
\subsection{Training with Counterfactuals}

\label{subsec:augmentation}
%\paragraph{Collection.}
Following \citet{kaushik2019learning}, we augment training sets with counterfactual examples,
\fixed{and demonstrate \sysname's usefulness by showing that its augmentations produce models that generalize better on \emph{some} datasets without hurting performance on others.}
In all experiments, we finetune \texttt{roberta-base} on datasets of $n$ original examples and $m$ counterfactuals, which are generated by \sysname (\maug) or crafted from scratch by humans (\mcad from \citet{kaushik2019learning}, only available for \nli). To distinguish the benefit of counterfactuals from that of just adding more data, we further add a baseline that uses $n+m$ original examples (\mcomp).
% examples from the original dataset
In addition to in-domain test set accuracy, we measure models' generalization on out-of-domain datasets, as well as contrast sets and challenge sets.
We also evaluate model capabilities with CheckList~\cite{checklist:acl20} for \sst and \qqp.
%As such datasets are unavailable for \qqp, we instead evaluate model capabilities with CheckList~\cite{checklist:acl20}.
Reported model performances are averaged across multiple data samples and random seeds (Appendix~\ref{appendix:data_collection}).


For \sst, we select random \sysname counterfactuals regardless of their labels, as long as an original $x$ has at least one $\xp$ that flips the label.
For \nli and \qqp, we observed in a pilot study that randomly chosen counterfactuals may not be more effective than the same amount of additional data.
We suspect that \sysname lacks domain knowledge and context for identifying critical perturbations, and therefore brings benefits redundant with pre-training~\cite{longpre2020effective}.
Thus, we use the slicing functions of \citet{chen2019slice} to find patterns of interest (\eg prepositions in \nli), and perturb those patterns by placing \texttt{[BLANK]}s on the matched spans. For example, \exinline{His surfboard is beneath him} becomes \exinline{His surfboard is \texttt{[BLANK]} him}, and \sysname generates counterfactuals such as \exinline{His surfboard is \swap{beneath}{next to} him.}


\textbf{Results.}
%Tables \ref{table:aug_sst} \& \ref{table:aug_nli} indicate that \sysname augmentation is effective for \sst and \nli: \maug maintains in-domain test accuracy while consistently improving generalization accuracy in various out-of-domain and challenge datasets. 
%On \nli, \sysname counterfactuals are as effective or more effective than counterfactuals created from scratch (\mcad).
%On \qqp ($n{=}20,000$, $m{=}1,636$), \maug also maintains in-domain accuracy when compared to \mcomp, while significantly lowering error rates (the absolute rate drops for at least $5$ points, with a relative difference of more than $10\%$) in $11$ out of $27$ tests that previously had high error rates (\sysname{} increases error rates in $2/27$ tests).
%\tofix{We aim to demonstrate the usefulness of Polyjuice’s augmentation by showing that it produces models that generalize better on \emph{some} datasets without hurting performance on others. }
Tables \ref{table:aug_sst}--\ref{table:aug_qqp} indicate that \sysname augmentation is effective in all tasks: \maug maintains in-domain accuracy while \emph{consistently} improving generalization accuracy in various out-of-domain and challenge sets. 
On \nli, \sysname counterfactuals are as effective or more effective than counterfactuals created from scratch (\mcad).
\sysname also improves on CheckList tests that previously had high error rates: 
it significantly lowers the error rates on 11 out of 27 \qqp tests,\footnote{The absolute error rate drops for at least $5$ points, with a relative difference of more than $10\%$.} making $2/27$ tests worse.
For \sst, it improves the model on 5 out of 15 tests, hurting 1.
\fixed{Notably, we obtain largest gains on challenge and contrast sets (\eg \texttt{Break} and \texttt{DNC} in Table~\ref{table:aug_nli}) or when the out-of-domain dataset is sufficiently different from the training domain (\eg Senti140 and SemEval in Table~\ref{table:aug_sst}). 
Here, we only report a low $m/n$ ratio (<10\% for NLI and QQP) to show that a small amount of augmentation is already beneficial. 
The results are similar for other combinations we explored (see Appendix~\ref{appendix:data_collection}), except when the ratio of counterfactual to original data was too high (\eg, $m=n$ may decrease vocabulary diversity or induce additional data bias, echoing~\cite{Khashabi2020MoreBF}).}


\subsection{Discussion}
\label{subsec:label_efficiency}
We show that \sysname counterfactuals are useful for evaluation, and more effective than additional (non-counterfactual) data for training in a variety of tasks. 
In contrast to prior work where humans generate counterfactuals from scratch, we only ask them to \emph{label} automatically generated ones, while still achieving similar or better results.

We believe our approach is more effective than manual creation (although both are beneficial): in terms of implementation effort, the process of just labeling counterfactuals is the same as labeling original examples, such that no additional annotator training or separate pipelines are required; in contrast, \citet{kaushik2019learning} set up two separate crowdsourcing tasks for creating and labeling the counterfactuals.
% Further, annotator effort is much lower, as evaluating examples is easier than creating them --- the median time for labeling one round (three $\xp$) was $30$ seconds, while \citet{kaushik2019learning} reported that workers spent ${\approx}4$ minutes per \nli revision (two $\xp$), \emph{prior} to quality validation.
% Even after we remove noisy annotators and labels and disregard non-fluent counterfactuals, we still collect one \nli $\xp$ used in Table~\ref{table:aug_nli} every 0.6 minutes.
Further, annotator effort is much lower, as evaluating examples is easier than creating them --- \citet{kaushik2019learning} report an average of ${\approx}2$ minutes per \nli counterfactual \emph{prior} to quality validation, while our median time was $10$ seconds per counterfactual. Even after our quality validation (removing noisy annotators, disregarding non-fluent counterfactuals), our rate for \nli is ${\approx}36$ seconds per counterfactual (used in Table~\ref{table:aug_nli}).

% the median time for labeling one round (three $\xp$) was $30$ seconds, while \citet{kaushik2019learning} reported that workers spent ${\approx}4$ minutes per \nli revision (two $\xp$), \emph{prior} to quality validation.
% Even after we remove noisy annotators and labels and disregard non-fluent counterfactuals, we still collect one \nli $\xp$ used in Table~\ref{table:aug_nli} every 0.6 minutes.
%1.6 $\xp$ used in Table~\ref{table:aug_nli} per minute.
%\nli revision (two $\xp$, translating to roughly 0.5 $\xp$ per minute), \emph{prior} to quality validation.
%Even after we remove noisy annotators and labels, disregard non-fluent counterfactuals, and apply aggressive post-filtering (\eg only keep $\xp$ that flip labels for contrast sets), we still obtain a suitable \nli $\xp$ in less than one minute.
%one \nli $\xp$ used in Table~\ref{table:aug_nli} every 0.6 minutes.
%we still obtain one suitable \nli $\xp$ used in Table~\ref{table:aug_nli}in 0.6 minute.
%It is also cheaper than labeling non-counterfactuals, as annotators only need to parse the corresponding perturbations, rather than the full instance~\cite{Khashabi2020MoreBF}.

In terms of the utility per counterfactual, manual creation and \sysname may be complementary. 
Human's creativity may be unreliable and incomplete on certain forms of counterfactuals~\cite{ribeiro2018semantically}, whereas \sysname can miss more complex or context-dependent changes, and would benefit from target perturbations that compensate for its lack of domain knowledge (in fact, targeted guidance is also helpful for human annotators~\cite{huang2020counterfactually}).
% such as changing the sentence structure
%not always emphasize on critical features if not appropriately controlled.
Thus, prior work suggests it may be important to mix both approaches~\cite{Khashabi2020MoreBF}.
\sysname's flexibility opens up possibilities for hybrids between human creation and human verification of targeted, machine-generated counterfactuals in future work.


%\emph{The necessity for human efforts.}
% Crowdworkers rated 82\% \dnli counterfactuals to be fluent, $70\%$ for \dqqp, and $75\%$ for \dsst.
% The number drops from \S\ref{subsec:intrinsic} due to the diversity selection, which forces certain unnatural combinations of \tagstrs and blank placements to be included.
% Moreover, we observe that 40\%--60\% $\xp$ do not flip the groundtruth labels.
% The imperfect filtering, and the complexity of class labels determines that human efforts are necessary. 

% However, rather than solely relying on humans, just \emph{labeling} counterfactuals can be equally useful.
% In fact, labeling is also cheaper, as (1) annotators are better at \emph{verifying} counterfactuals than manually \emph{generating} them~\cite{ribeiro2018sear}, and (2) as they label the three variations, annotators only need to focus on the reference example and the corresponding perturbed phrases, rather than re-parsing the full instance for each label they submit~\cite{Khashabi2020MoreBF}.
% As such, the median time for labeling one round (three $\xp$) is 30 seconds.
% Even with aggressive filtering on data quality and groundturth labels (for collecting contrast sets), we still obtain one high-quality, label-flipping $\xp$ in 1--2.5 minutes.
% This is 40\%--75\% more efficient than manual creation:
%Because \sysname does not control the change of groundtruth label, when generating contrast sets (\S\ref{subsec:contrast_set}), we remove $\xp$ that maintains the label (40\% in \nli and 63\% in \sst).
%Combined with the filtering on noisy workers and fluency, 
%the average time for collecting one label flipping $\xp$ is 1--2.5 minutes, which is still 40\%--75\% more efficient than manual creation:
% \citet{kaushik2019learning} reported that workers spent roughly 5 minutes to revise an IMDb review and 4 minutes an \nli sentence, prior to quality-based validation.

% In other words, \sysname helps to lift the burden of manual rewrites, and reallocate human efforts to the \emph{easier} but \emph{more critical} steps of verification and labeling.
% As a result, our human-\sysname team collects data more effectively.

