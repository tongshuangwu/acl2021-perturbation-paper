\section{Discussion}
\label{sec:discuss}

We separate the automatic counterfactual generation into an \emph{application-agnostic} generation step, and an \emph{application-specific} selection step.
As a result, all applications have access to the same pool counterfactuals generated by \sysname, which are fluent, diverse, and close to the original sentence.
With additional selection methods on the relationship between $x$ and $\xp$, \sysname supports various downstream tasks, including counterfactual data augmentation, contrast set generation, as well as counterfactual explanation.
We have made \sysname available, and plan to opensource the implementations for the selection strategies.
Below, we discuss promising future work:


\emph{More balanced sentence pairs and sampling.}
Because most existing paired datasets highlight just the most contrasting patterns, the \sysname model inevitably inherits frequent pairs from them.
For example, its lexical changes are more likely to return \swap{man}{woman} than \swap{man}{child}.
To further improve the vocabulary diversity, we can emphasize more on \emph{finding naturally occurring sentence pairs in non-paired dataset}. 
We only used heuristics on text overlaps to broaden some \tagstrs, but methods like synactic tree editing or knowledge graph distances can also be useful.

\emph{Supporting existing counterfactual reasoning tools.}
Besides independent use cases, future work can also explore plugging \sysname into existing tools that involve counterfactual generation.
%For example, featurizing counterfactuals into Snorkel labeling functions~\cite{ratner2017snorkel} helps collect the training and evaluation data more efficiently (similar to \S\ref{subsec:global_exp}).
For example, domain experts can create perturbation tests more easily if \sysname serves as an additional building block for error analysis~\cite{wu2019errudite} or model testing~\cite{checklist:acl20}.


\emph{Human-in-the-loop counterfactual generation.}
Humans can add more values than placing blanks and labeling counterfactuals
For example, humans can increase the diversity of the covered patterns, if they are asked to come up with more counterfactuals that are different from those covered automatically.
In tasks that require context-dependent counterfactual generations --- a difficult task for the sentence-based model alone --- human annotators can also initiate seed counterfactuals for \sysname to extend.
For example, to perturb multi-hop question answering instances~\cite{gardner2020contrast}, the human can start by adding the compositional reasoning steps \emph{related to} the corresponding paragraph, so the model can further perturb around it.