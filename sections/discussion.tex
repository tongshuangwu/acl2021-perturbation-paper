\section{Conclusion and Future Work}
\label{sec:discuss}

We propose \sysname, a general-purpose generator that produces fluent and diverse counterfactuals, allowing for control over the kinds and locations of perturbations. 
% With \emph{task-agnostic} controls over perturbation types and locations, \sysname produces general-purpose counterfactuals that are fluent, diverse, close to the original instance.
With simple, \emph{task-specific} selection heuristics, \sysname supports various downstream tasks on different domains, including counterfactual data augmentation, contrast set generation, counterfactual explanation, and error analysis.

While \sysname is broadly applicable, it is not bias-free: control codes are pre-defined and certainly not exhaustive, and the model is fine-tuned on a collection of paired datasets where certain perturbations are more or less likely (\eg we observe that words with negative sentiment tend to be slightly more likely than positive ones in some contexts). 
Collecting naturally occurring counterfactuals is an important area of future research, as is the development of generators that allow for control even without \emph{a-priori} control codes.

\fixed{
Besides improving the generators, further work is needed to improve the value of counterfactuals.
For example, while \sysname shows consistent gains across tasks in data augmentation, the improvements on some datasets are not as significant.
This aligns with observations in prior work that even manual counterfactuals can be marginally beneficial~\cite{kaushik2019learning, huang2020counterfactually}, possibly because the original data is already diverse enough, or the perturbed signal in counterfactuals is too subtle to affect the model (\eg when only a single word is changed in a long sentence.)
%too implicit to grasp.
We hope to perform more thorough experiments on tuning the amount and the distribution of counterfactual augmentation, as well as other ways of incorporating counterfactuals, such as having explicit terms in the loss function for contrasting counterfactuals with original data~\cite{teney2020learning}, or other forms of contrastive learning.
}

\fixed{Although} our applications all involved people, the human-\sysname collaboration in labeling and explanations could benefit from richer interaction mechanisms. 
We believe \sysname motivates future research on more expressive forms of counterfactual training, where users generate counterfactuals together with \sysname, and label counterfactual \emph{patterns} rather than individual instances. 
Similarly, interactive explanations and analysis are exciting directions, especially as we develop new ways of selecting, presenting, and aggregating counterfactuals for various analysis objectives.
%to coordinate with various analysis objectives of humans.
%aggregating counterfactuals and their effects. 
Having noted these opportunities, we believe \sysname is already a powerful tool for counterfactual reasoning, in particular for tasks where people are directly involved. 
\sysname is opensource, and available at \modelurl.
%\sysname will be open-sourced, together with the selection strategies in this paper.


%We create counterfactuals through \emph{task-agnostic} generation, and \emph{task-specific} selection.
%As a result, all applications have access to the same pool of counterfactuals generated by \sysname, which are fluent, diverse, and close to the original sentence.
%With additional selection methods, \sysname supports various downstream tasks, including counterfactual data augmentation, contrast set generation, counterfactual explanation, and error analysis.
% The success of \sysname results from three key designs: (1) the finetuning on paired sentences, (2) the predefined \tagstrs, and (3) the blank placement.
% Conversely, \sysname relies on the quality of the datasets, and its coverage on counterfactuals relies on the careful design of \tagstrs.
% We indeed observe that \sysname inherits salient contrast patterns from existing paired datasets.
% For example, it is more likely to change \remove{man} to \add{woman} than \add{child}.
% To mitigate the potential dataset bias, future work should explore naturally occurring sentence pairs 

% The keys to the success of \sysname include its finetuning on paired datasets, 
% One of the keys to \sysname is finetuning on paired sentences, which is the backbone for learning diverse blank placements and \tagstrs.
% However, the learning inevitably comes with bias, and \sysname inherits salient contrast patterns from existing paired datasets.
% For example, it is more likely to change \remove{man} to \add{woman} than \add{child}.
% To further improve vocabulary diversity, we encourage the field to define more rigorous methods for finding naturally occurring sentence pairs in non-paired dataset.
% We relied on heuristics of text overlap, but syntactic tree editing or knowledge graph distances may be promising.

% In most of our demonstrated applications, \sysname pairs with a person, taking over the generation step and extending it beyond human rewrites or simple paraphrasing. 
% As such, people can effectively focus on labeling and analyzing the counterfactuals.
% We believe the human-model team is a promising framework for future work, which may involve designing control codes that better suit user needs (\eg for error analysis hypothesis testing), implementing collaborative frameworks where people and the model complement each others' generations, etc.
% On the other hand, we would also like to tune the filtering and constraints, to verify that \sysname can also advance automated use cases like adversarial attacks.

% We plan to open source both the \sysname model and our counterfactual selection methods.
