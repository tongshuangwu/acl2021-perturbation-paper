\section{Conclusion and Future Work}
\label{sec:discuss}

We propose \sysname, a general-purpose counterfactual generator that produces fluent and diverse counterfactuals, allowing for control over the kinds and locations of perturbations. 
% With \emph{task-agnostic} controls over perturbation types and locations, \sysname produces general-purpose counterfactuals that are fluent, diverse, close to the original instance.
With simple, \emph{task-specific} selection heuristics, we show how \sysname supports various downstream tasks on different domains, including counterfactual data augmentation, contrast set generation, counterfactual explanation, and error analysis.

While \sysname is broadly applicable, it is not bias-free: control codes are pre-defined and certainly not exhaustive, and the model is finetuned on a collection of paired datasets where certain perturbations are more or less likely (\eg we observe that words with negative sentiment tend to be slightly more likely than words with positive sentiment in some contexts) . We think the sourcing of datasets with naturally occurring counterfactuals is an important area of future research, as well as the development of generators that allow for control even without \emph{a-priori} control codes.

Though all of our applications involved humans in some way, the human-\sysname collaboration in labeling and explanations is very simple, not allowing for much interactivity. 
We believe tools like \sysname motivate future research on more expressive forms of counterfactual training, where users label counterfactual \emph{patterns} rather than individual instances, potentially in an interactive rather than static manner. 
Similarly, interactive explanations and analysis are exciting directions of future work, especially as we develop new ways of aggregating groups of counterfactuals and their effects. Having noted these opportunities, we believe that \sysname is already a powerful tool for counterfactual reasoning in NLP, in particular for tasks where humans are involved. \sysname will be open-sourced, together with the selection strategies used in this paper.


%We create counterfactuals through \emph{task-agnostic} generation, and \emph{task-specific} selection.
%As a result, all applications have access to the same pool of counterfactuals generated by \sysname, which are fluent, diverse, and close to the original sentence.
%With additional selection methods, \sysname supports various downstream tasks, including counterfactual data augmentation, contrast set generation, counterfactual explanation, and error analysis.
% The success of \sysname results from three key designs: (1) the finetuning on paired sentences, (2) the predefined \tagstrs, and (3) the blank placement.
% Conversely, \sysname relies on the quality of the datasets, and its coverage on counterfactuals relies on the careful design of \tagstrs.
% We indeed observe that \sysname inherits salient contrast patterns from existing paired datasets.
% For example, it is more likely to change \remove{man} to \add{woman} than \add{child}.
% To mitigate the potential dataset bias, future work should explore naturally occurring sentence pairs 

% The keys to the success of \sysname include its finetuning on paired datasets, 
% One of the keys to \sysname is finetuning on paired sentences, which is the backbone for learning diverse blank placements and \tagstrs.
% However, the learning inevitably comes with bias, and \sysname inherits salient contrast patterns from existing paired datasets.
% For example, it is more likely to change \remove{man} to \add{woman} than \add{child}.
% To further improve vocabulary diversity, we encourage the field to define more rigorous methods for finding naturally occurring sentence pairs in non-paired dataset.
% We relied on heuristics of text overlap, but syntactic tree editing or knowledge graph distances may be promising.

% In most of our demonstrated applications, \sysname pairs with a person, taking over the generation step and extending it beyond human rewrites or simple paraphrasing. 
% As such, people can effectively focus on labeling and analyzing the counterfactuals.
% We believe the human-model team is a promising framework for future work, which may involve designing control codes that better suit user needs (\eg for error analysis hypothesis testing), implementing collaborative frameworks where people and the model complement each others' generations, etc.
% On the other hand, we would also like to tune the filtering and constraints, to verify that \sysname can also advance automated use cases like adversarial attacks.

% We plan to open source both the \sysname model and our counterfactual selection methods.




\begin{comment}
%In tasks that require context-dependent generations --- difficult for the sentence-based model alone --- human annotators can also initiate seed counterfactuals for \sysname to extend.
%For example, to perturb question answering instances~\cite{gardner2020contrast}, the human can add the compositional reasoning steps \emph{related to} the corresponding paragraph for \sysname to perturb around.

\emph{More balanced sentence pairs and sampling.}
\sysname inherits some most most salient contrasts pairs from existing paired datasets.
For example, it is more likely to change \remove{man} to \add{woman} than \add{child}.
To further improve vocabulary diversity, we can emphasize more on \emph{finding naturally occurring sentence pairs in non-paired dataset}.
We used heuristics on text overlaps to broaden some \tagstrs, but methods like syntactic tree editing or knowledge graph distances can also be useful.



We plan to opensource both the \sysname model and the implementation of the selection methods.




\emph{Supporting existing counterfactual reasoning tools.}
Besides independent use cases, \sysname may also be plugged into existing tools that involve counterfactual generation.
%featurizing counterfactuals into Snorkel labeling functions~\cite{ratner2017snorkel} helps collect the training and evaluation data more efficiently (similar to \S\ref{subsec:global_exp}).
For example, domain experts can create perturbation tests more easily if \sysname serves as an additional building block for model testing~\cite{checklist:acl20}.
Or, they may translate $\relation{\xp}$ into labeling functions for data programming~\cite{ratner2017snorkel}.


\emph{Human-in-the-loop counterfactual generation.}
Humans can add more values than placing blanks and labeling counterfactuals, \eg supply missing perturbation patterns after seeing \sysname's generations.
In tasks that require context-dependent generations --- difficult for the sentence-based model alone --- human annotators can also initiate seed counterfactuals for \sysname to extend.
For example, to perturb question answering instances~\cite{gardner2020contrast}, the human can add the compositional reasoning steps \emph{related to} the corresponding paragraph for \sysname to perturb around.
\end{comment}