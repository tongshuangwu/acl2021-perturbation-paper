\section{Survey of Perturbation Applications}

Table~\ref{table:perturb_application} summarizes the perturbation applications and their generation strategies.

%\begin{comment}
\begin{table*}
\small
\centering
%\setlength{\tabcolsep}{3.5pt}
\begin{tabular}{r c c p{0.55\linewidth}}
\toprule
\textbf{Application} & \textbf{$y = \hat{y}$}? & $f(x) = f(\xp)$? & \textbf{Strategies} \\ 
\midrule
Data augmentation & \cmark & \qmark & 
    lexical~\cite{Wu2019ConditionalBC, Wei2019EDAED, Kumar2020DataAU}\newline
    paraphrasing~\cite{iyyer2018adversarial} \newline
    perturbation functions~\cite{ratner2017snorkel}
\\\midrule
Counterfactual data aug. & \xmark & \qmark & 
    manual~\cite{kaushik2019learning} \newline
\\\midrule
Adversarial attack & \cmark & \xmark & 
    lexical~\cite{alzantot2018generating, garg2020bae, li-etal-2020-bert-attack, morris2020textattack, tan2020s, jin2020bert, ebrahimi2017hotflip, Zhang2019GeneratingFA, Jia2019CertifiedRT} \newline
    template~\cite{jiang2019avoiding}\newline
    insert~\cite{Song2020UniversalAA}
\\\midrule
Contrast set & \xmark & \qmark & 
    manual~\cite{li2020linguistically} \newline
    templates~\cite{li2020linguistically}
\\\midrule
Challenge sets & \qmark & \qmark & 
    lexical (heuristic)~\cite{kaushik2019learning, naik2018stress} \newline
    paraphrasing~\cite{Kavumba2019WhenCP} \newline
    templates~\cite{Geiger2019PosingFG, kaushik2019learning, nie2019analyzing, mccoy2019right}
\\\midrule
Model analysis & \qmark & \qmark & 
    Template~\cite{Goodwin2020ProbingLS}\newline
    Perturbation functions~\cite{wu2019errudite, bowman-etal-2015-large}
\\\midrule
Explanations & \qmark & \qmark & 
    lexical~\cite{hase2020evaluating, vig2020causal, kang2020counterfactual} \newline
    % probably also lexical
    lexical (mask)~\cite{ramon2019counterfactual, ribeiro2018anchors, }
\\
\bottomrule
\end{tabular}

\caption{Paper survey on the perturbation applications.}
\label{table:perturb_application}
\end{table*}
%\end{comment}
