%\begin{comment}
\begin{table*}[t]
\small
\centering
\setlength{\tabcolsep}{4pt}
\begin{tabular}{@{}lrrrrrrrrr@{}}
\toprule
\textbf{Dataset} & \textbf{\ctrltag{negation}} & \textbf{\ctrltag{quantifier}} & \textbf{\ctrltag{leixcal}} & \textbf{\ctrltag{resemantic}} & \textbf{\ctrltag{insert}} & \textbf{\ctrltag{delete}} & \textbf{\ctrltag{restructure}} & \textbf{\ctrltag{shuffle}} & \emph{\ctrltag{global}} \\ 
\midrule
        CAD &      3,456 &         457 &    10,650 &        4,634 &    2,169 &    2,162 &          234 &       84 &    3,756 \\
   Contrast &       336 &         436 &     1,607 &        1,291 &     589 &     586 &          275 &      149 &     877 \\
       HANS &        50 &           0 &        0 &           0 &    3,926 &    3,926 &          494 &     1,602 &       2 \\
    ParaNMT &      2,797 &         825 &    10,000 &       10000 &    6,442 &    6,205 &         5,136 &     1,417 &   10,000 \\
       PAWS &        81 &        1,815 &    10,000 &       10000 &    3,630 &    3,403 &         4,551 &    10,000 &   10,000 \\
 WinoGrande &      3,011 &          94 &    10,000 &        6,927 &     120 &     124 &          453 &       65 &    3184 \\
    \emph{Crawled} &         0 &           0 &     5,000 &           0 &    5,000 &    5,000 &            0 &      108 &    5,000 \\
      \textbf{Total} &      9,731 &        3,627 &    47,257 &       32,852 &   21,876 &   21,406 &        11,143 &    13,425 &   32,819 \\
\bottomrule
\vspace{-15pt}
\end{tabular}
\caption{The datasets used for finetuning the GPT-2 generation model, and the \tagstr distributions.}
\label{table:gpt_train_stats}
\vspace{-10pt}
\end{table*}



\section{GPT-2 as Counterfactual Generator}
\label{appendix:train_data}

\subsection{Training data collection}


We combine the following NLP datasets for finetuning \sysname.
To balance the distribution (shown in Table~\ref{table:gpt_train_stats}), for each dataset, we extract \tagstrs from all the data pairs, and randomly sample up to 10,000 instances per \tagstr.

\textbf{Contrast set.}
Authors of 10 existing NLP dataset each manually perturbed 100--1,000 instances to change the gold label, so to inspect a model's local decision boundary~\cite{gardner2020contrast}.
The perturbation patterns vary based on the tasks and the annotators, allowing us to learn diverse strategies.
To make sure we can use the contrast set to evaluate the \sst model, we excluded the IMDb movie review from the training.\footnote{Similarly, though \qqp would be a potentially interesting dataset for training \sysname, we omitted it so \qqp can be used in our evaluation.}
%\wts{Re-train the model with other contrast sets.}


\textbf{Counterfactually-augmented data (CAD).}
\citet{kaushik2019learning} crowdsourced counterfactuals for IMDb movie review (1.7k) and SNLI (6.6k).
Similar to the contrast set, CAD's perturbation patterns vary based on the task, but can especially contribute to \ctrltag{negation}.
We split the movie review paragraphs into paired sentences to match the sentence length of other datasets.


\textbf{WinoGrande} is a large-scale dataset of 44k instances for testing common sense problems~\cite{sakaguchi2019winogrande}.
It contains sentences that differ only by one trigger word (\eg one noun), making it most suitable for learning lexical exchanges.

\textbf{ParaNMT-50M} contains 50 million English-English sentential paraphrase pairs, covering various domains and styles of text, as well as different sentence structures~\cite{wieting2017paranmt}. 

\textbf{PAWS}~\cite{zhang2019paws} contains 108k paraphrasing and non-paraphrasing pairs, created through controlled word swapping and back translation, best demonstrating the \ctrltag{shuffle} and \ctrltag{restructure} strategies.


\textbf{HANS}~\cite{mccoy2019right}, a challenge set for NLI, contains 10k pairs of premises and hypotheses created based on 10 heavily fallible syntactic templates, and therefore compensates rarer structural changes that may be missed by PAWS.


\textbf{Crawled} 
We additionally crawl naturally occurring sentence pairs from non-paired datasets like SQuAD~\cite{rajpurkar-etal-2016-squad} to boost some specific patterns and increase lexical diversity. 
We estimate \emph{close} pairs using edit distance, and broadly include those with less than 60\% editing, so to include negative samples (see below.)
To exclude non-counterfactuals (\eg \exinline{how do I not be} and \exinline{how do I recover it} can be incorrectly considered asv\ctrltag{negation} pairs), we only crawl pairs for the most determined patterns: \ctrltag{lexical}, \ctrltag{insert}, \ctrltag{delete}, and \ctrltag{shuffle}.
%Among them, we further filter the pairs using \tagstrs (see the section below).
%Still, unrelated examples 


\subsection{Training Prompts \& Parameters}

Given a sentence pair, we use the two sentences interchangeably as $x$ and $\xp$ to learn the \tagstrs both ways.
For a $(x, \xp)$, we compute its primary \tagstr based on linguistic features like part-of-speech tagging or dependency trees, and blank out the changed subtrees in $\xp$.
For example, \ctrltag{negation} occurs when we observe changes on negation modifiers or specific words like ``supposedly'', and \ctrltag{shuffle} occurs when we have overlaps between tokens deleted and added.
When multiple changes occur, we label it with the primary \tagstr, which most significantly changes the semantic meaning on the corresponding subphrase.
In Figure~\ref{fig:blank}A, we use \ctrltag{negation}, as \swap{great}{not great} is more significant than \swap{kids}{children}.
If we cannot identify the \tagstr for a $(x, \xp)$ or if the editing distance is too large, we denote it with \ctrltag{global} and use it as a negative training sample.
Importantly, to allow flexible blanking at generation time, we generate multiple training prompts from one $(x, \xp)$, with different blanking strategies: (1) just the changed tokens, (2) the associated parsing structures, (3) the merged changes, and (4) the entire sentence (some examples are in Figure~\ref{fig:blank}).
%As a result, we form up to four unique training prompts given one $(x, \xp)$ pair 

%max_log_prob_diff
%With the interchangeable orders and the blanks, we generate 657,144 training prompts from 191,415 sentence pairs.
We use the data to finetune an off-the-shelf GPT-2 model from \citet{Wolf2019HuggingFacesTS}, but any LM can potentially be used.
%\wts{Add the training hyperparameters; And maybe eventually move them to appendix.}
We finetuned the model for 10 epochs with an initial learning rate 5e-5, a batch size of 8, and a sequence length of 120.
We select the best epoch based on the evaluation loss on a holdout set of size 5,000.
The training took around 8 hours on two Titan's RTX.


\subsection{Intrinsic Evaluations}
\label{appendix:intrinsic}






\subsubsection{Controllability with Ablation Studies}
\label{appendix:ablation_control}

We finetune another GPT-2 model with training prompts that \emph{do not} contain \tagstrs (called \emph{\sysname-a}), and quantify the impact of the \tagstrshorts through an ablation study.
For each \tagstr, we compare the \emph{control success rate} of \sysname and \sysname-a on 250 prompts (from 100 unique original sentences).
For each prompt, we generate counterfactuals through beam search (beam $=10$), and recompute the \tagstrshorts on the top three returns.
We deem the control successful if at least one of the three recomputed \tagstrshorts matches the input (though in \sysname-a, we only measure whether the \tagstrshort naturally occurs in the uncontrolled generation.)
The success rate increases by $28.4\% \pm 18.2\%$ across all \tagstrs, ranging from \ctrltag{quantifier} (increasing 8\%, from 40.4\% to 48.4\%) to \ctrltag{insert} (64.1\%, from 13.5\% to 78.6\%).
%\ctrltag{lexical} has the smallest increment \tofix{from $93\%$ to $100\%$}, mostly because \emph{\sysname-a} tend to frequently replace words.
%\ctrltag{insert} is the most impactful \tagstrshort (from \tofix{$12\%$ to $100\%$}) --- \sysname-a rarely insert additional clues on its own.

There are three common failure cases for the \tagstrshorts:
%Note that all the prompts used are guaranteed to allow the corresponding \tagstrs, and the \tagstrshorts can be less effective on more general prompts:
(1) The dual manipulation from the \tagstrs and the blanks can conflict, \eg \exinline{a dog is embraced by a \texttt{[BLANK]}} would not respond to \ctrltag{negation}.
(2) $x$ does not have a corresponding pattern. \ctrltag{shuffle} is not applicable when the sentence has only one adjective or noun (\eg \exinline{the movie is good}).
(3) Certain pattern is very prominent that it dominates the generation probability, \eg the model tends to perturb the quantifier ``two'' in \exinline{two dogs are running}, regardless of the \tagstrshort.
In the ablation study, we filtered out prompts that fell under cases 1 and 2.







\begin{table}[tb]
\small
    \centering
    %\setlength{\tabcolsep}{1.3pt}
    \begin{tabular}{@{}lccc@{}}
    \toprule
    \multirow{2}{*}{Model} & Diversity & \multicolumn{2}{c}{Closeness} \\
    \cmidrule(lr){2-2}
    \cmidrule(lr){3-4}
    & Self-BLEU $\uparrow$ & Semantic $\downarrow$ & Tree edit $\downarrow$ \\
    % \cmidrule{2-4}
    \midrule
    \emph{\sysname} & \textbf{0.82} & \textbf{0.37} & \textbf{2.02} \\
    Masked-LM & 0.66 & \textbf{0.27} & \textbf{1.89} \\
    PPLM-BoW & \textbf{0.82} & 0.65 & 7.65 \\
    \bottomrule
    \end{tabular}
    \vspace{-2.5mm}
    \caption{The intrinsic metrics comparing \sysname with Masked-LM and PPLM-BoW. 
    $\uparrow$ (or $\downarrow$) indicates whether the metric should be maximized (or minimized).
    As expected, \sysname counterfactuals are \emph{closer} to the original instance than PPLM-BoW, and more \emph{diverse} than the Masked-LM ones.}
    \vspace{-3mm}
    \label{table:intrinsic}
\end{table}

\subsubsection{Closeness, Diversity, Fluency}

Similar to \citet{madaan2020generate}, we verify that the \sysname generations are fluent (``plausible'' in their case) through human evaluations (\S\ref{subsec:label_efficiency}).
We also quantify the diversity and closeness by comparing it with baseline models.

For a given $x$ and its counterfactuals $\hat{\xset}$, we approximate \emph{diversity} using self-BLEU~\cite{malandrakis-etal-2019-controlled, zhu2018texygen} within the generated counterfactual set $\hat{\xset}$.
The higher the BLEU, the more lexically different the generated phrases are to each other. 
Meanwhile, \emph{closeness} is measured by the average distance $x$ and every $\xp \in \hat{\xset}$, both semantically (using a sentence similarity model~\cite{reimers-2019-sentence-bert}) and syntactically (tree edit distance~\cite{zhang1989simple})

We use similar baselines as \citet{madaan2020generate}: 
(1) \emph{Masked-LM}, where we blank certain parts of $x$ (by randomly placing up to three \texttt{[MASK]} tokens), and ask RoBERTa to fill in the blank.
We rely on the CheckList implementation~\cite{checklist:acl20}, as it allows filling in multiple blanks at once through beam search. 
(2) \emph{PPLM-BoW}~\cite{Dathathri2020Plug}, a model that uses bag-of-words to control the generation.
We use the first two words of $x$ as the input context (prompt), limit the length of the generation to be similar to $x$, and apply their default condition ``positive words.''
As the model generates \emph{arbitrary text} that do not depend on $x$, we agree with \citet{madaan2020generate} that PPLM-BoW should not satisfy the \emph{closeness} requirement.

We randomly select 120 instances from \qqp, \dnli, and \dsst (40 per dataset), and generate 10 $\xp$ per $x$ using each of the three generators.
As shown in Table~\ref{table:intrinsic}, \sysname achieves compatible diversity with PPLM-BoW, and compatible \emph{closeness} with Masked-LM, achieving a balance between both.
Ideally, we would also like to compare \sysname with our concurrent work GYC~\cite{madaan2020generate} and MiCE~\cite{ross2020explaining}.
%Inspired by style transfer~\cite{yang2018unsupervised} and controlled text generation, GYC performs the perturbation on the latent space of the input $x$.
%Meanwhile, MiCE uses a two-step framework to generate conterfactual explanations, with the generator being T5~\cite{JMLR:v21:20-074} finetuned on the task-specific dataset.
%As mentioned in \S\ref{sec:relate}, both generators focus on flipping the class label of a given $x$.
Unfortunately, both require extensive implementation or finetuning, and has yet to be opensourced.

%Still, we believe that our infilling (\texttt{BLANK}) structure will improve the \emph{closeness} over the unconstrained GYC (called content and syntactic preservation in their case).
%We also hypothesize that \sysname has better (or at least compatible) diversity than GYC. 
%Per their intended use case --- model testing and debiasing --- the control functions in GYC are driven by the predictor-of-interest (\eg sentiment classifier, NER tagger). 
%As a result, most of the GYC changes seem to focus on features deemed important by the classifier (\eg \exinline{I am very disappointed with the service} is changed by \swap{disappointed}{pleased}, \swap{disappointed}{happy}, \swap{disappointed with}{pleased to get a good}).


